\frontmatter
\chapter{前言}
Android的影响力 
为什么要学习Android 

\section*{本书的特点}
\begin{itemize}
\item 通过一个逐步完善的实例展示Android开发的方方面面
\item 始终让读者有强烈的参与感
\end{itemize}
\section*{如何阅读本书} 



在阅读本书时,经常需要了解两个版本之间的差别,常见的方法有两种:
\begin{itemize}
\item 命令行方式:在终端窗口执行svn diff -r 版本1:版本2。更多的参数可以执行svn help diff
\item 借助于Eclipse等IDE工具,请参考不同工具的说明文档。
\end{itemize}

本书的附录~\ref{chap:code_history}也列出了代码演变的过程供读者参考。

另外,本书旨在引导读者快速入门,因此在不影响对基本概念的理解和掌握的前提下,对于技术的细节做了有意的忽略和删减。不过,在合适的地方本书也给出了了解更多细节的链接或者参考。
本书在几乎所有章节也列举了常见的设计技巧,帮助读者更快的掌握常见的Android程序设计方法。

\section*{你适合阅读本书吗}
本书假设你有如下的基础:
\begin{itemize}
\item Java的基本概念
\item XML的基础知识,最好知道一些XHTML的知识
\item 如果有WEB程序设计经验,无论使用什么语言,就更好了
\end{itemize}

事实上,很多人是被很多Android的专业书籍开篇的第一章吓退的!那些书往往列举了Android所包含的种种技术,初学者一看就头大,再看可能连勇气都都丢掉了!这绝不是Android的初衷!本书的目的就是让初学者在一个贯穿始终的例子中,轻松的掌握Android的编程基本原理和方法。
\section*{本书的体例}
\subsection*{本书的印刷约定}
\begin{tabular}{|l|l|l|}
\hline
字体 & 意义 & 示例 \\
        \hline
        \textsl{AbCd123}斜体 & 文件名、路径名、域名等 & ls -l \textsl{filename}\\
            \hline
            \textbf{AbCd123}加粗 & 在终端输入的命令等 & subaochen\_desktop\% \textbf{su} \\
                \hline
                等宽字体 & 示例代码、代码片段等 & publc class MyClass... \\
                    \hline

                    \end{tabular}

                    \subsection*{类的命名方式}
                    本书中,类的命名遵循如下的原则:
                    \begin{itemize}
                    \item 接口采用自然的命名方式,比如UserManager接口。
                    \item 接口的实现类是在接口的名字后面增加Bean,比如UserManager接口的某个实现是DbUserManagerBean,以强化“一切都是Bean”的理念。
                    \end{itemize}
                    \section*{联系我们}
                    您可以通过我的博客获得最新的消息:http://dz.sdut.edu.cn/blog/subaochen,本书所有的源代码都可以从https://github.com/subaochen/android-tutorial获得,您可以在本书的不同章节看到如何获得源代码的相关提示。建议您遵循这些提示下载不同阶段的源代码反复把玩。热情欢迎大家参与到android-tutorial的开发中来,您可以通过邮箱subaochen@126.com联系作者以获得提交代码的权限~\label{参与openeshop开发的方法}。


                    \mainmatter
